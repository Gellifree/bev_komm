\documentclass[a4paper,12pt]{article}
\usepackage[T1]{fontenc}
\PassOptionsToPackage{defaults=hu-min}{magyar.ldf}
\usepackage[magyar]{babel}
\usepackage[a4paper]{geometry}
\geometry{margin=2cm}
\usepackage{listings}
\usepackage{graphicx}
\usepackage[table]{xcolor}
\usepackage{fancyhdr}
\usepackage{listingsutf8}

\begin{document}
	\pagestyle{fancy}
	\fancyhf{}
	\fancyhead[LE,RO]{\normalfont\normalsize\thepage}
	\fancyhead[RE]{\nouppercase{\sffamily\small\leftmark}}
	\fancyhead[LO]{\nouppercase{\sffamily\small\rightmark}}
	
	\title{{\Large Beadandó feladat} \\Bevezetés a kommunikációelméletbe  \\ {\small (LBG\_KM731K2)} \\[1cm] {\huge Eric Berne: Emberi játszmák}}
	
	\author{Kovács Norbert és Szabó Bence}
	
	\date{\today}
	\maketitle
	
	\newpage
	
	\section{Bevezetés}
	\subsection*{Eric Berne}
	Eric Berne, kanadai pszichiáter, aki leginkább a \textit{tranzakcióanalízis} kidolgozójaként, és az \textit{Emberi játszmák} című könyve révén vált híressé.
	Számomra Berne legérdekesebb és legfontosabb gondolata szerint a szervezetnek ingerlésre van szüksége a megfelelő funkcionáláshoz. Itt felmerül egy nagyon kifejező gondolat, hogy az ingerlés a legfontosabb, nem pedig a kellemes inger. Emiatt a gondolat miatt szerettem volna belevetni magam a munkájába, és elemezni az Emberi játszmák című könyvét.
	
	\subsection{Társas érintkezés}
	Az első, és legfontosabb gondolat a könyv első fejezetében, hogy a hanyagolás, és érzelmi nélkülözés leépüléshez vezet. Könnyedén párhuzamba hozható az ingeréhség, és a valódi táplálék éhség. 

	\begin{flushright}
		{\footnotesize 		,,Ha az agytörzs retikuláris aktiváló rendszerét nem éri elegendő inger,\\ akkor az idegsejtekben -- legalábbis közvetve -- elsorvadási folyamat indulhat meg. Előfordulhat, hogy ez a folyamat a rosszul tápláltság másodlagos hatása, de a hiányos táplálkozás maga is lehet az apátia terméke''}
	\end{flushright}

	Számomra ebből a legfontosabb tanulság, hogy közvetve, vagy közvetett módon mindenféleképpen egy leépülés követi az ingerszegénységet, és komoly egészségügyi és fizikai következményekkel kell számolnunk.
	
	Kifejező, hogy sok terminológiát áthelyezhetünk szinte egyenesen a táplálkozás területéről. Az ingeréhséget a táplálékéhséggel, a túltápláltságot a túlingerléssel stb. A könyv megemlít bizonyos fogalmakat amiknél nem fedi le a párhuzam másik oldalát, ami nagyon kíváncsivá tett.
	Megpróbáltam  megfogalmazni saját magam ezeket a párhuzamokat, de arra jutottam, hogy egy-egy kifejezésre, több párhuzam is hozható, és legtöbbször személyes, egyéni tapasztalat miatt fűzöm össze. 
	\begin{itemize}
		\item A megemlített \textit{ínyencséget} először a saját olvasási szokásaimmal hoztam párhuzamba, miszerint science fictiont, csak egy írótol vagyok ,,hajlandó'' olvasni. \\
		De ínyencségnek fogalmaznám meg azt is, amikor egy \textbf{konkrét}, \textbf{speciális} elismerési formára vágyik az ember egy szerepkörben. Pl.: ,,nagyon jó vezető vagy'', ,,nagyszerű oktató vagy''.
		\item Az aszkézist, a ki nem mondott igazsággal hoznám párhuzamba. Főleg egy olyan közegben ahol nagy ellenérzéseket, tompítást vált ki minden negatív gondolat, vagy ,,rossz'' dologra való figyelemfelhívás.
		\item A torkosságot a véget nem érő, élvezetes beszélgetések vágyával kötném össze, amikor olyan személlyel kommunikálunk akinek a véleménye hasonlatos a miénkkel, és ,,feltartjuk'' egymást. \\
		Számomra itt szintén megjelenik párhuzamként a ,,túltolod'' kifejezés. Az élet különböző területein való túlteljesítés véleményem szerint torkosság. Célja szerintem a túlzott elismerés kiváltása.
	\end{itemize}
	Ezen a ponton úgy gondolom, hogy a hozott párhuzamok teljesen személyes tapasztalatokra épülnek, és az hogy egy pontra több párhuzam is megjelent, annak a jele hogy nem elég általános az értelmezésem. Ettől függetlenül kifejezetten érdekes gondolatnak tartom, hogy az étkezési szokásainkat ilyen szerepkörben is lehet használni.
	
	El szoktuk fogadni a gondolatot hogy ,,nem tehetek róla hogy éhes vagyok, ennem kell''. Viszont nagyon nehéz kimondani hogy ,,törődésre van szükségem, szeressetek''. Pedig mind a kettő érzés, eszerint a megközelítés szerint rajtunk kívülálló szükségletünk, még is máshogy kezeljük.
	
	Ennek a feloldása nálam ott történt meg, amikor megemlítjük a rejtett, jelképes gondozási formákat. Mivel nehéz kijelentenünk, vagy megfogalmaznunk ezeket a szükségleteinket, \textit{kompromisszumokat} kötünk a társadalommal. Megjelennek a jelképes gondozási formák. A könyv példája szerint, egy színésznek hetente akár 100 ,,simogatásra'' is szüksége van, amíg egy tudósnak évente elég egy simogatás is a megfelelő szakmai vezetőtől.
	
	\begin{flushright}
		{\footnotesize 		,,A \textit{simogatást} az intim fizikai kapcsolat általános fogalmának értelmében használjuk; a gyakorlatban ez különféle formákat  ölthet.\\ \dots \\ 
		A \textit{simogatás}  fogalmát, jelentésbővítéssel, köznyelvileg minden olyan aktus jelölésére alkalmazhatjuk, amely egy másik személy jelenlétét nyugtázza.''}
	\end{flushright}

	Ennek a fejezetnek az utolsó gondolata valahol nagyon meglepett. A konklúzió hogy a gyengéd törődés és az áramütés is pozitív hatású, amíg a hanyagolás káros.
	Úgy gondolom hogy ez egy olyan erős alapja a bántalmazó kapcsolatoknak, ami kissé új fénybe borítja a bántalmazott szerepkört. Ha ennyire mélyen rejtőzik az, hogy a bántalmazás is jobb mint a hanyagolás, és megjelenik a ,,legalább foglalkozik velem'' kifejezés, valójában bármilyen aspektusból hibáztatható-e a bántalmazott? (Sokszor megjelenik az a motívum, hogy a bántalmazott hagyja magát, és ő is tehet róla) Egy sokkal vitathatóbb kérdés, hogy mennyire hibáztatható a bántalmazó ebben a szerepkörben. Lehetséges, hogy sok esetben a bántalmazó ezt törődési formának érzékeli. Sokszor elhangozhat az a mondat, hogy ,,addig jó amíg veszekszek, mert az azt jelenti hogy még érdekel''. Az intenzív veszekedés, és nem vita, alapvetően romboló jellegű a véleményem szerint, mégis sokan ezt elfogadható eszköznek érzik a törődés kifejezésére. 
	

	\subsection{Az idő strukturálása}
	Mindenkinek feladata és felelőssége az ébrenlét órát strukturálni, hogy eleget tegyen a struktúra éhségének. Berne megfogalmazásában az idő strukturálását nevezhetjük programozásnak, és ezt három csoportra bontja: \textit{anyagi}, \textit{társadalmi} és \textit{egyéni}.
	
	A könyvben elhangzik egy hasonlat, a hajóépítés dinamikájáról. A hajóépítők csoportjának minden társas érintkezésüket alá kell rendelniük az előre megszabott munkafolyamatoknak, mert csak így haladhat előre a hajó építése. Az \textit{anyagi programozás} forrása a külső valósággal való megbirkózás, lényegében az adatfeldolgozás az alapja.
	
	A \textit{társadalmi programozás} egy sokkal érdekesebb csoport. Ide a hagyományos, rituális vagy fél rituális érintkezéseket soroljuk. Ezeket az érintkezéseket nevezhetjük jó modornak, amit szüleinktől tanulunk meg, és nagyon változatosak lehetnek. 
	
	\begin{flushright}
		{\footnotesize 		,,Helyi, ősi hagyomány bátorítja  vagy  tiltja,  lehet-e  böfögni  az  étkezésnél,  lehet-e érdeklődést tanúsítani  más felesége  iránt. E két sajátságos 	tranzakció  között például fordított kölcsönösségi viszony figyelhető meg: rendszerint nem tanácsos az asszonynép után érdeklődni  ott,   ahol  az  emberek    böfögnek  étkezés  után; olyan  helyeken  viszont,  ahol  lehet az asszonynép után érdeklődni, ott nem tanácsos étkezéskor böfögni.''}
	\end{flushright}

	Berne szerint amikor két ember közelebbről megismeri egymást, egyre több \textit{egyéni programozás} csúszik be. Ezek afféle incidensek, viszont egyértelműen mintát követnek, csoportosíthatóak és osztályozhatóak.
	
	Ezek az időtöltések és játszmák, a valódi intimitás pótlékai. A valódi intimitás akkor kezdődik el, amikor az egyéni programozás intenzívebbé válik mint a társadalmi szabályozás. 
	
	\begin{flushright}
		{\footnotesize 	Az intimitás az egyetlen tökéletesen kielégítő válasz az ingeréhségre,  az  elismeréséhségre  és  a  struktúraéhségre. Prototípusa a szerelmi egyesülés.}
	\end{flushright}

	Az idő strukturálásának célja az unalom, magány elkerülése. Csoportban is lehet valaki magányos, viszont egy csoportban több lehetőségünk nyílik az idő strukturálására. A csoportosulás tagjainak célja, hogy a többiekkel lebonyolított tranzakciókból a lehető legtöbb kielégüléshez jussanak. Fontos, hogy a kielégülés szót nehéz egyértelműen értelmezni, mert sok ilyen tranzakció önpusztító jellegű, ezért Berne a nyereség, vagy előny szavakat használja ezentúl.
	
	Az idő strukturálását  feloszthatjuk rituálékra, időtöltésekre, játszmákra, intimitásra és tevékenységekre. Úgy a társas érintkezésből származó nyereségeink, feszültségcsökkenésre, ártalmas helyzetek elkerülésére, a simogatásunk megszerzésére, és az egyensúly fenntartására oszthatóak.
	
	A legfontosabb hogy ezeket sokkal hasznosabb előnyként vizsgálni, mint védekezésként. Ugyanis a legjobb védekezés, ha nem folytatunk semmilyen tranzakciót, és a védekezés nem is fedi le teljesen az említett szempontokat.

	\section{Játszmaelemzés}
	\subsection{Strukturális elemzés}
	
	Ebben a fejezetben az \textit{én-állapotok} fogalmába kapunk részletesebb betekintést. Köznyelvi megnevezésükön: szülői, felnőtti, gyermeki én-állapot. Mindenki egy adott pillanatban, valamelyik én-állapotot engedi megnyilvánulni.
	
	Először, én úgy gondoltam, hogy egy egészséges személyiség esetén a Felnőtti én-állapotnak kell uralkodnia, viszont rosszul hittem. Minden én állapotnak kritikusan fontos szerepe van a hétköznapokban.
	\begin{itemize}
		\item A Gyermekben intuíciók, kreativitás, alkotókészség, spontán hajtóerők és örömök jelennek meg. Én ezt úgy értelmezem, hogy a bennünk rejlő gyermek éli meg örömeinket, talál ki innovatív megoldásokat.
		\item A Felnőtt a fennmaradáshoz elengedhetetlen. Számításokat, adatelemzést végez, a segítségével birkózunk meg a világ problémáival, és feladatainkkal. A Felnőtt szerzi meg a tapasztalatokat, például arról mi nyújt örömöt vagy kudarcot. A Felnőttnek még az is feladata hogy szabályozza a Szülő és Gyermek tevékenységét, és tárgyilagosan közvetítsen. Számomra a megfontolt, mérlegelt objektív szemléletet képviseli a Felnőtt.
		\item A Szülő elsődleges funkciója, a konkrét, valóságos gyermekek nevelése. Ezenkívül bizonyos folyamatokat, és válaszokat automatikussá tesz. Ezzel rengeteg ,,számítást'', időt és energiát spórol meg a Felnőttnek. Számos dolog megszokás kérdése, és nem kell a Felnőttnek apróbb ügyekben is ,,intézkednie'', a rutinügyeket a Szülőre bízza.
	\end{itemize}
	Számomra kifejezetten érdekes, hogy ezeknek az én állapotoknak a szerepköre kiegyensúlyozott kell hogy legyen, és problémák akkor jelenhetnek meg, amikor valamelyik én-állapot túlnyomóan uralkodik. Fontos, hogy a ,,gyerekes'' szónak olyan negatív mellékzöngéi vannak, ami miatt ez esetben nem is használunk. Úgy gondolom, hogy amikor valaki a Gyermek én-állapotot engedi érvényesülni, egy olyan szituációban, ahol a Felnőtt, analitikus én-állapotra lenne szükség, ekkor fogalmazódik meg bennünk valakiről hogy gyerekes. Viszont látnunk kell, hogy a Gyermek ugyan olyan nagy szerepkörrel rendelkezik mint a másik két én-állapot.
	
	Ezek az én-állapotok teljesen elkülönülnek egymástól, sokszor egymással ütköznek is, mégis egy családon belül véleményem szerint ezek egy láncolatot alkothatnak. Tudjuk hogy hatással vagyunk egymásra, és ,,alakulunk'' szeretteinkhez, barátainkhoz. A valódi szülő-gyermek kapcsolat esetén, véleményem szerint hozható párhuzam az én állapotok között. (A szüleink Felnőtt, és a mi Szülő én-állapotunk szerintem szoros kapcsolatban állnak)
	
	A Gyermeki én állapot megfogalmazása szerint, mindig voltunk fiatalabbak magunknál, ezért van olyan korábbi pillanat, ahonnan hozhatunk magunkkal rögzült maradványokat, amik előhozhatóak. 
	Mindenki képes objektíven szemlélni a valóságot (beleértve a gyermekeket, értelmi fogyatékosakat, vagy szkizofréneket is), tehát mindenkiben lakozik egy Felnőtt. És mindannyiunknak megvannak a saját szüleink vagy szülőpótlékaink, ahonnan újratermelhetjük a szülők/szülőpótlékok én-állapotait. 
	
	\subsection{Tranzakcionális elemzés}

\end{document}