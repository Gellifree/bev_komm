\documentclass[a4paper,12pt]{article}
\usepackage[T1]{fontenc}
\PassOptionsToPackage{defaults=hu-min}{magyar.ldf}
\usepackage[magyar]{babel}
\usepackage[a4paper]{geometry}
\geometry{margin=2cm}
\usepackage{listings}
\usepackage{graphicx}
\usepackage[table]{xcolor}
\usepackage{fancyhdr}
\usepackage{listingsutf8}

\begin{document}
	\pagestyle{fancy}
	\fancyhf{}
	\fancyhead[LE,RO]{\normalfont\normalsize\thepage}
	\fancyhead[RE]{\nouppercase{\sffamily\small\leftmark}}
	\fancyhead[LO]{\nouppercase{\sffamily\small\rightmark}}
	
	\title{{\Large Beadandó feladat} \\Bevezetés a kommunikációelméletbe  \\ {\small (LBG\_KM731K2)} \\[1cm] {\huge Eric Berne: Emberi játszmák}}
	
	\author{Kovács Norbert és Szabó Bence}
	
	\date{\today}
	\maketitle
	
	\newpage
	\section*{Eric Berne}
	Eric Berne, kanadai pszichiáter , aki leginkább a \textit{tranzakcióanalízis} kidolgozójaként, és az \textit{Emberi játszmák} című könyve révén vált híressé.
	Számomra Berne legérdekesebb és legfontosabb gondolata szerint a szervezetnek ingerlésre van szüksége a megfelelő funkcionáláshoz. Itt felmerül egy nagyon kifejező gondolat, hogy az ingerlés a legfontosabb, nem pedig a kellemes inger. Emiatt a gondolat miatt szerettem volna belevetni magam a munkájába, és elemezni az Emberi játszmák című könyvét.
	
	\section{Társas érintkezés}
	Az első, és legfontosabb gondolat a könyv első fejezetében, hogy a hanyagolás, és érzelmi nélkülözés leépüléshez vezet. Könnyedén párhuzamba hozható az ingeréhség, és a valódi táplálék éhség. 

	\begin{flushright}
		,,Ha az agytörzs retikuláris aktiváló rendszerét nem éri elegendő inger,\\ akkor az idegsejtekben -- legalábbis közvetve -- elsorvadási folyamat indulhat meg. Előfordulhat, hogy ez a folyamat a rosszul tápláltság másodlagos hatása, de a hiányos táplálkozás maga is lehet az apátia terméke''
	\end{flushright}

	Számomra ebből a legfontosabb tanulság, hogy közvetve, vagy közvetett módon mindenféleképpen egy leépülés követi az ingerszegénységet, és komoly egészségügyi és fizikai következményekkel kell számolnunk.
	
	Kifejező, hogy sok terminológiát áthelyezhetünk szinte egyenesen a táplálkozás területéről. Az ingeréhséget a táplálékéhséggel, a túltápláltságot a túlingerléssel stb. A könyv megemlít bizonyos fogalmakat amiknél nem fedi le a párhuzam másik oldalát, ami nagyon kíváncsivá tett.
	Megpróbáltam  megfogalmazni saját magam ezeket a párhuzamokat, de arra jutottam, hogy egy-egy kifejezésre, több párhuzam is hozható, és legtöbbször személyes, egyéni tapasztalat miatt fűzöm össze. 
	\begin{itemize}
		\item A megemlített \textit{ínyencséget} először a saját olvasási szokásaimmal hoztam párhuzamba, miszerint science fictiont, csak egy írótol vagyok ,,hajlandó'' olvasni. \\
		De ínyencségnek fogalmaznám meg azt is, amikor egy \textbf{konkrét}, \textbf{speciális} elismerési formára vágyik az ember egy szerepkörben. Pl.: ,,nagyon jó vezető vagy'', ,,nagyszerű oktató vagy''.
		\item Az aszkézist, a ki nem mondott igazsággal hoznám párhuzamba. Főleg egy olyan közegben ahol nagy ellenérzéseket vált ki minden negatív gondolat, vagy ,,rosz'' dologra való figyelemfelhívás.
		\item A torkosságot a véget nem érő, élvezetes beszélgetések vágyával kötném össze, amikor olyan személlyel kommunikálunk akinek a véleménye hasonlatos a miénkkel, és ,,feltartjuk'' egymást. \\
		Számomra itt szintén megjelenik párhuzamként a ,,túltolod'' kifejezés. Az élet különböző területein való túlteljesítés véleményem szerint torkosság. Célja szerintem a túlzott elismerés kiváltása.
	\end{itemize}
	Ezen a ponton úgy gondolom, hogy a hozott párhuzamok teljesen személyes tapasztalatokra épülnek, és az hogy egy pontra több párhuzam is megjelent, annak a jele hogy nem elég általános az értelmezésem. Ettől függetlenül kifejezetten érdekes gondolatnak tartom, hogy az étkezési szokásainkat ilyen szerepkörben is lehet használni.
	
	El szoktuk fogadni a gondolatot hogy ,,nem tehetek róla hogy éhes vagyok, ennem kell''. Viszont nagyon nehéz kimondani hogy ,,törődésre van szükségem, szeressetek''. Pedig mind a kettő érzés, eszerint a megközelítés szerint rajtunk kívülálló szükségletünk, még is máshogy kezeljük.
	
	Ennek a feloldása nálam ott történt meg, amikor megemlítjük a rejtett, jelképes gondozási formákat. Mivel nehezen megközelíthető kijelentenünk, vagy megfogalmaznunk ezeket a szükségleteinket, \textit{kompromisszumokat} kötünk a társadalommal. Megjelennek a jelképes gondozási formák. A könyv példája szerint, egy színésznek hetente akár 100 ,,simogatásra'' is szüksége van, amíg egy tudósnak évente elég egy simogatás is a megfelelő szakmai vezetőtől.
	
	\begin{flushright}
		,,A \textit{simogatást} az intim fizikai kapcsolat általános fogalmának értelmében használjuk; a gyakorlatban ez különféle formákat  ölthet.\\ \dots \\ 
		A \textit{simogatás}  fogalmát, jelentésbővítéssel, köznyelvileg minden olyan aktus jelölésére alkalmazhatjuk, amely egy másik személy jelenlétét nyugtázza.''
	\end{flushright}

	Ennek a fejezetnek az utolsó gondolata valahol nagyon meglepett. A konklúzió hogy a gyengéd törődés és az áramütés is pozitív hatású, amíg a hanyagolás káros.
	Úgy gondolom hogy ez egy olyan erős alapja a bántalmazó kapcsolatoknak, ami kissé új fénybe borítja a bántalmazott szerepkört. Ha ennyire mélyen rejtőzik az, hogy a bántalmazás is jobb mint a hanyagolás, és megjelenik a ,,legalább foglalkozik velem'' kifejezés, valójában bármilyen aspektusból hibáztatható-e a bántalmazott? (Sokszor megjelenik az a motívum, hogy a bántalmazott hagyja magát, és ő is tehet róla) Egy sokkal vitathatóbb kérdés, hogy mennyire hibáztatható a bántalmazó ebben a szerepkörben. Lehetséges, hogy sok esetben a bántalmazó ezt törődési formának érzékeli. Sokszor elhangozhat az a mondat, hogy ,,addig jó amíg veszekszek, mert az azt jelenti hogy még érdekel''. Az intenzív veszekedés, és nem vita, alapvetően romboló jellegű a véleményem szerint, mégis sokan ezt elfogadható eszköznek érzik a törődés kifejezésére. 
	

\end{document}